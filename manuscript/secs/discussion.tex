The identification of novel biomarkers to diagnose and guide treatment of CRC is a priority of contemporary cancer genomics research \citep{Sveen2019}.
Biomarkers associated with immunogenic potential are of particular relevance given the emergence of immune checkpoint inhibition as an effective treatment for advanced disease.
In this study, we performed the largest transcriptome-wide, unbiased assessment of unannotated human endogenous retroviral elements in colorectal tissue to date.
We found several unique proviral loci with tumor specific over expression and many other loci with relatively high expression in healthy samples.
The tumor specific HERV sequences we identified discriminate tumor from healthy tissue and histologically normal tissue adjacent to neoplastic mucosa, which highlights their possible utility in screening and surveillance.
Furthermore, if the identified sequences are translated, they could encode neoantigens with immunogenic potential that could serve as targets of bispecific antibody therapies in addition to tumor specific biomarkers.

Remarkably, one of the HERV loci with highest expression in cancer was Xp22

This study also demonstrated the frequency of HERV expression in healthy colorectal tissue.
HERVs may play a role in maintaining healthy epithelium
Healthy overexpression ... functional role
This finding was contrary to our hypothesis and also undermined an alternative explanation of our
This was an important finding because it was contrary to our hypothesis and because it provided
First, it was contrary to our hypothesis.
Second, it provides some

Limitations of this study included a potential for batch effects from different studies and age confounder

These results are important because
This study is the largest to date to investigate transcriptome-wide effects of CRC on adjacent, histologically normal mucosa using RNA-seq.
The results are the first to demonstrate the consistent overexpression of 20 tumor-associated genes in histologically normal tissue sampled adjacent to tumors.
Remarkably, genes contributing to established oncogenic pathways, such as epidermal growth fact
