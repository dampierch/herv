\subsection*{Sample selection}
Bulk RNA-seq from a curated set of 1,139 flash frozen colorectal tissue samples was re-processed and analyzed for the presence of unannotated HERV transcripts \citep{Dampier2020}.
For discovery of associations between tissue phenotypes and HERV expression, paired end sequencing from 833 colon biopsies was used.
Of those biopsies, 462 were from healthy, 62 were from tumor adjacent normal, and 309 were from tumor tissue.
This set of samples is referred to as the discovery cohort.

For validation of discovered associations, two independent cohorts were used.
The first (independent cohort \#1) included single end sequencing from 276 colon biopsies of which 12 were healthy, 56 were tumor adjacent, and 208 were tumor.
The second (independent cohort \#2) included paired end sequencing from 30 pairs of tumor and matched adjacent normal biopsies (i.e. 15 samples of both tissue phenotypes paired by subject).
An important characteristic of the discovery cohort and independent cohort \#1 was the independence of all tumor and tumor adjacent samples.
To permit valid inference from regular linear models, only one tissue phenotype was used per subject.

Samples were originally gathered from publicly available RNA-seq datasets hosted on the Genomic Data Commons and Sequence Read Archive as well as from a recently generated dataset of healthy mucosal biopsies obtained during screening colonoscopies at the Catalan Institute of Oncology in Barcelona, Spain \citep{Dampier2020, DiezObrero2020}.
A complete description of sample selection and cohort construction for the curated dataset is available in Dampier et al. (2020) \citep{Dampier2020}.
RNA-seq reads for all samples were stored and analyzed in FASTQ format.
For samples downloaded in BAM format, Biobambam2 v2.0.87 \citep{Tischler2014} was used to convert reads to FASTQ format.

\subsection*{Human endogenous retrovirus annotation}
HERVs were identified using the genomic coordinates (GRCh38) and feature names of exons and spliced transcripts of HERV elements from a recently published human cancer transcriptome annotated with the GENCODE v24 basic gene model \citep{Frankish2018} and a model of genome-wide repeat elements \citep{Attig2019}.
The cancer transcriptome was assembled from RNA-seq reads of 24 subjects from each of 32 cancer types obtained from The Cancer Genome Atlas (TCGA), as previously described \citep{Attig2019}.
Names and coordinates of repeat elements were generated using RepeatMasker \citep{Smit2015} configured with nhmmer \citep{Wheeler2013} in sensitive mode using the Dfam 2.0 library (v150923) \citep{Hubley2015}, as previously described \citep{Attig2017}.
The method relies on profile hidden Markov models representing known human repeat families to classify repeat elements throughout the genome.

Previously unannotated HERV genes were selected for the current study in multiple steps.
A complete gene transfer format (GTF) \citep{GTF} annotation of the repeat element inclusive cancer transcriptome was downloaded from Attig et al. (2019) \citep{Attig2019}.
The annotation was parsed with a custom Python script to produce a new annotation in BED12 \citep{BED12} format.
Transcripts were selected based on three criteria: i) overlap of at least one exon with a long terminal repeat element, ii) classification of at least one exon as HERV derived, and iii) absence of any exon overlapping an annotated GENCODE gene.
All transcripts labeled with the same repeat element identifier (i.e. repeat element class along with genomic coordinates) were considered isoforms of the same HERV locus.
Genomic coordinates were extracted for all constituent exons of selected transcripts to permit inference of splice sites and exclusion of intronic sequences.
For display purposes, repeat element identifier genomic coordinates were converted to cytobands based on the cytoBand table downloaded from the UCSC Genome Browser \citep{Furey2003, Navarro2020}.

\subsection*{Transcript quantification}
Transcript quantification was performed with Salmon v1.2.1 \citep{Patro2017} in mapping-based mode with the \verb|--gcBias| and \verb|--validateMappings| flags set.
Inputs to the Salmon \verb|quant| command were RNA-seq reads in FASTQ format and a custom Salmon index.
To create the Salmon index, a custom reference transcriptome was built from a list of GENCODE v34  transcript sequences \citep{Frankish2018} and the list of selected HERV transcript sequences.
GENCODE sequences were downloaded from the GENCODE FTP site \citep{GENCODE-transcripts}.
HERV sequences were extracted from the GRCh38 reference genome \citep{ENCODE-GRCh38} using the BEDTools \verb|getfasta| command \citep{Quinlan2010} with the \verb|-split| option and the custom BED12 annotation of selected HERV transcripts mentioned above.
Following the recommendation of Salmon's authors \citep{SalmonDecoys}, contig sequences from the GRCh38 reference genome were subsequently added to the target transcript sequences as decoy sequences to generate a decoy-aware transcriptome for indexing.

\subsection*{Proviral locus and protein coding gene expression analysis}
Associations between HERV locus expression and tissue phenotype were tested using the \emph{Bioconductor} suite of genomics analysis packages \citep{bioc} in the R statistical programming language \citep{R}.
Specifically, \emph{tximport} \citep{Soneson2015} was used to extract transcript length and abundance estimates from Salmon output and summarize them by HERV locus and GENCODE gene for input into \emph{DESeq2} \citep{Love2014}.
Count matrices of HERV loci and GENCODE genes were prefiltered to select loci and genes with a count of at least one read in at least half of samples in each cohort.
Factors to normalize counts for sequencing depth and transcript length were calculated with \emph{DESeq2}.
Latent factors of non-biological variation were estimated with \emph{sva} \citep{sva} as described in the Supplementary Methods.
Loci and genes of interest for association testing were selected in several steps.

First, counts of all loci and genes were transformed with the variance stabilizing transformation of \emph{DESeq2} and adjusted for batch effects with \emph{limma} \citep{Ritchie2015}.
Second, adjusted counts of all annotated genes were filtered to retain only protein coding genes.
Third, the median and standard deviation of protein coding gene expression were calculated separately for each tissue phenotype in each cohort.
Finally, adjusted counts of HERV loci were filtered to retain only HERVs expressed within a single standard deviation of median protein coding gene expression in at least one tissue phenotype.

Counts of all HERV loci and protein coding genes with sufficient expression were modeled with phenotype and latent factors as explanatory variables using \emph{DESeq2}.
Significant associations between gene count and tissue phenotype were identified by testing whether the absolute value of the effect of phenotype was at least two fold change for a given gene.
Multiple hypothesis testing correction was performed using a false discovery rate of 5\%.

All figures were produced with \emph{ggplot2} \citep{Wickham2016} as described in the Supplementary Methods.
