\textbf{Background}:
Transcripts from novel genomic features such as endogenous retroviruses have potential utility as biomarkers or therapeutic targets, and their characterization could help improve detection and treatment of colorectal cancer.
To identify uncharacterized transcripts for possible diagnostic and therapeutic applications, we examined cancer specific over-expression of unannotated endogenous retroviral mRNA in a large cohort of colorectal transcriptomes.

\textbf{Methods}:
Bulk RNA sequencing reads from human primary colorectal tissue, including 474 healthy, 133 tumor adjacent normal, and 532 tumor samples, were obtained from public sources and from our own data and re-processed using a custom reference transcriptome.
Transcripts were quantified using Salmon with GENCODE transcripts combined with novel endogenous retroviral transcripts from a recently described cancer transcriptome.
Associations between transcript expression and tissue phenotype were determined using \emph{DESeq2} in the R statistical programming language.

\textbf{Results}:
Of 5,025 unannotated proviral loci queried in our discovery analysis, 150 were expressed within one standard deviation of median protein coding gene expression in at least one tissue phenotype.
Tumor specific over expression was observed in 11 of the 150 loci tested (\emph{P} adjusted < 0.05), and seven loci were validated in an independent cohort.

\textbf{Conclusions}:
In the largest unbiased assessment of unannotated human endogenous retroviral transcript expression in colorectal tissue to date, we identified over 100 proviral loci expressed at levels similar to canonical protein coding genes and seven that could be novel biomarkers of colorectal cancer or targets of immunotherapy.
