\textbf{Background}:
Microsatellite unstable, mismatch repair deficient metastatic cancers can be treated with immune checkpoint inhibitors, but new therapeutic approaches are needed to extend the benefits of immune modulators to the majority of patients with microsatellite stable tumors.
Cancer specific over expression of human endogenous retroviral elements could be exploited to enhance immune infiltration of microsatellite stable colorectal cancer and render these tumors more susceptible to immune checkpoint inhibition.
To identify novel targets for immune mediated cancer therapies, we tested for cancer specific over expression of endogenous retroviral transcripts in a large cohort of colorectal transcriptomes.

\textbf{Methods}:
A recently curated RNA sequencing dataset composed of 1,139 human primary colorectal tissue samples divided into three cohorts was re-processed using a novel annotation of repeat element transcripts to quantify levels of repeat element expression in colorectal transcriptomes.
A total of NUMBER healthy, NUMBER tumor-adjacent normal, and NUMBER tumor samples were profiled with short read sequencing.
Transcripts were quantified with Salmon using the GENCODE v36 transcriptome augmented to include unannotated exons from long terminal repeat elements classified as human endogenous retroviruses (HERVs) in the Dfam database.
Associations between transcript expression and tissue phenotype were tested using DESeq2 in the R statistical language.

\textbf{Results}:
Out of 5,025 unannotated HERV genes measured in 834 samples, including NUMBER healthy, NUMBER tumor-adjacent normal, and NUMBER tumor samples, 150 were expressed within one standard deviation of median protein coding gene expression in at least one tissue phenotype.
Over half of the 150 HERV genes with moderate expression were in the H (26) or K (51) classes.
Tumor specific over expression was observed in 11 of the 150 HERV genes tested.

\textbf{Conclusions}: The conclusions go here.
