\textbf{Figure 1.} Summary of unannotated human endogenous retroviral sequences found in colorectal transcriptomes. \\
a) Barplot demonstrating numbers of human endogenous retroviral elements considered in this study.
Numbers of elements are shown on the y-axis, and categories of retroviral elements considered are shown on the x-axis. \\
b) Barplot demonstrating numbers of human endogenous retroviral elements in each of several retroviral groups expressed within one standard deviation of the median of canonical protein coding genes.
Numbers of elements are shown on the y-axis, and groups of retroviral elements are shown on the x-axis. \\
c) Histogram demonstrating distributions of expression levels of canonical protein coding genes (i.e. GENCODE genes) and detected human endogenous retroviral loci across all samples in the discovery cohort. \\
d) Histogram demonstrating distributions of expression levels of canonical protein coding genes (i.e. GENCODE genes) and detected human endogenous retroviral loci across tumor samples in the discovery cohort. \\

\textbf{Figure 2.} Differential expression of unannotated human endogenous retroviral loci in colorectal transcriptomes. \\
a) Scatterplot (i.e. volcano plot) demonstrating relationship between $log2(fold change)$ and $-log10(\emph{P})$ for all human endogenous retroviral loci and protein coding genes tested in the discovery cohort with baseline expression set by the healthy phenotype.
$log2(fold change)$ is shrunken (see Supplementary Methods).
\emph{P} values are not corrected for multiple hypothesis testing.
Grey points are either protein coding genes or non-differentially expressed proviral loci.
Blue points are proviral loci negatively associated with the tumor phenotype.
Red points are proviral loci positively associated with the tumor phenotype.
\emph{TUM} = tumor, \emph{HLT} = healthy. \\
b) Scatterplot (i.e. volcano plot) demonstrating relationship between $log2(fold change)$ and $-log10(\emph{P})$ for all human endogenous retroviral loci and protein coding genes tested in the discovery cohort with baseline expression set by the tumor adjacent phenotype.
$log2(fold change)$ is shrunken (see Supplementary Methods).
\emph{P} values are not corrected for multiple hypothesis testing.
Grey points are either protein coding genes or non-differentially expressed proviral loci.
Blue points are proviral loci negatively associated with the tumor phenotype.
Red points are proviral loci positively associated with the tumor phenotype.
\emph{NAT} = normal adjacent to tumor. \\
c) Boxplots demonstrating expression levels by phenotype of 11 proviral loci positively associated with the tumor phenotype in the discovery cohort (\emph{P} adjusted < 0.05). \\

\textbf{Figure 3.} Global estimates of unannotated human endogenous retroviral expression across tissue phenotypes. \\
Heatmap demonstrating standardized expression levels (i.e. Z-scores) of human endogenous retroviral loci tested in the discovery cohort across all samples in the cohort.
Proviruses (N = 150) are listed on the y-axis and ordered according to group.
Group O is an acronym for "Other".
Samples (N = 833) are listed on the x-axis and ordered according to phenotype.
Darker panels indicate higher standardized expression.
