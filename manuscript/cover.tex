\documentclass[11pt,letterpaper]{article}  %% 3500 words for ASO research article

    %% load packages
    \usepackage[margin=1in]{geometry}         %% document margins
    \usepackage[utf8]{inputenc}
    \usepackage{setspace}                     %% to get \doublespacing command
    \usepackage{authblk}                      %% to get \affil command
    \usepackage{parskip}                      %% space between paragraphs
    \usepackage{fontspec}                     %% for \setmainfont command, will need xelatex to compile Arial

    %% general document formatting commands
    \pagestyle{plain}                         %% page numbers in middle
    %\linespread{1.67}                        %% baseline linespread = 1.2 (1.2*1.67~2)
    \setmainfont{Arial}                       %% https://latex.org/forum/viewtopic.php?t=25998

\begin{document}

January 24, 2021 \\
\\
Kelly M. McMasters, MD, PhD \\
Editor-in-Chief \\
Annals of Surgical Oncology \\
Professor and Chair, Department of Surgery \\
University of Louisville \\
Louisville, Kentucky \\

Dear Dr. McMasters, \\

We respectfully submit our manuscript, \textbf{Human Endogenous Retroviral Transcripts Show Potential as Novel, Tumor Specific Biomarkers for Colorectal Cancer}, for consideration for publication as an original research article in \emph{Annals of Surgical Oncology}.
This manuscript has not been previously published in any form, although a limited part of the data was presented in a parallel session at the Society of Surgical Oncology Annual Meeting in 2020.
This manuscript is not under consideration for publication elsewhere in any form.

Transcripts from novel genomic features such as endogenous retroviruses have potential utility as biomarkers or therapeutic targets, and their characterization could help improve detection and treatment of colorectal cancer.
To identify uncharacterized transcripts for possible diagnostic and therapeutic applications, we examined cancer specific over-expression of unannotated endogenous retroviral mRNA in a large cohort of colorectal transcriptomes.
We identified over 100 proviral loci expressed at levels similar to canonical protein coding genes and seven that could be novel biomarkers of colorectal cancer or targets of immunotherapy.

We believe the results of our study would be of interest to your readers and would appreciate the opportunity to publish these findings in \emph{Annals of Surgical Oncology}. Thank you for considering our manuscript for publication. \\

Sincerely,\\
\\
\\
Christopher H. Dampier, MD \\
Research Fellow \\
Center for Public Health Genomics \\
University of Virginia \\
Charlottesville, Virginia \\
Email: chd5n@virginia.edu \\
Twitter: @Dampier\_CH \\

Graham Casey, PhD \\
Professor, Department of Public Health Sciences \\
Center for Public Health Genomics \\
University of Virginia \\
Charlottesville, Virginia \\
Email: gc8r@virginia.edu

\thispagestyle{empty}
\end{document}
