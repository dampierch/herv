\documentclass[10pt,letterpaper]{article}

    %% general document formatting
    \usepackage[margin=1.5in]{geometry}
    \usepackage[utf8]{inputenc}
    \usepackage{authblk}
    \usepackage{bibentry}
    \usepackage{parskip}

    \title{Potential Novel Immunotarget Revealed in Mega Analysis of Colorectal Transcriptomes}

    \author[1]{Christopher H. Dampier}
    \author[2]{David F. Grabski}
    \author[1]{Matthew Devall}
    \author[3]{Virginia Diez-Obrero}
    \author[3]{Ferran Moratalla-Navarro}
    \author[4]{David Rekosh}
    \author[4]{Marie-Louise Hammarskjold}
    \author[5]{Sara K. Rasmussen}
    \author[3]{Victor Moreno}
    \author[1]{Graham Casey}

    \affil[1]{Center for Public Health Genomics, University of Virginia, Charlottesville, Virginia, USA}
    \affil[2]{Department of Surgery, University of Virginia, Charlottesville, Virginia, USA}
    \affil[3]{Catalan Institute of Oncology, Barcelona, Spain}
    \affil[4]{Department of Microbiology, Immunology and Cancer Biology, University of Virginia, Charlottesville, Virginia, USA}
    \affil[5]{Department of Surgery, Seattle Children's, Seattle, Washington, USA}

    %% redefine maketitle function to customize title formatting
    %https://tex.stackexchange.com/questions/85343/left-align-abstract-title-and-authors
    %https://en.wikibooks.org/wiki/LaTeX/Fonts#Sizing_text
    \makeatletter
    \renewcommand{\maketitle}{
        \begingroup
            \setlength{\parindent}{0pt}
            \begin{flushleft}
                \LARGE\textbf{\@title}
                \newline
                \newline
                \normalsize\@author
            \end{flushleft}
        \endgroup
    }
    \makeatother

\begin{document}

\maketitle

\section*{Abstract}
\textbf{Background}: The background goes here.

\textbf{Methods}: The methods go here.

\textbf{Results}: The results go here.

\textbf{Conclusions}: The conclusions go here.

\section*{Introduction}

Colorectal cancer (CRC) is a common malignancy in the United States.
Early detection and effective treatment result in a 5-year relative survival of 90\% for patients diagnosed with local disease.
However, over a fifth of new cases are diagnosed with metastatic disease, which has a 5-year relative survival of only 14\% \cite{SEER2020}.
To improve overall survival of patients with distant disease, new therapeutic approaches are needed.

Immunotherapy in CRC is a promising new treatment option for patients with distant disease.
Immune checkpoint inhibition was shown to achieve durable responses with programmed cell death protein 1 (PD-1) antibody monotherapy and combination therapy with antibodies for PD-1 and cytotoxic T-lymphocyte-associated
protein 4 (CTLA-4) \cite{Le2015, Overman2018}.
However, response to immune checkpoint inhibition was limited to the subset of tumors with mismatch repair defficiency and high microsatellite instability, which constitute approximately 4\% of all metastatic CRC cases \cite{Ganesh2019}.
Strategies to improve tumor immunogenicity and increase cytotoxic T cell infiltration are required to help extend the benefit of immune therapies to the other 96\% of metastatic CRC.
Epitopes derived from endogenous retroviral peptides expressed in cancers may offer a solution.

Human endogenous retroviruses (HERVs) are proviral elements integrated into the human genome under conditions that render them incompetent of replication or infection.
Residual HERV elements in contemporary human genomes are the consequence of retroviral infections of germ cells millions of years ago.
Since then, and like the rest of the human genome, HERV elements have accumulated point mutations and segmental deletions.
These alterations in HERV sequences disrupt assembly of functional viral machinery.
Nevertheless, some HERV sequences can still be transcribed and translated, and a subset are processed as endogenous antigens and presented on major histocompatibility complex (MHC) class I molecules \cite{Boller1997, Rooney2015}.
In healthy tissues, DNA methylation and histone modification can silence transcription of HERV genes, but epigenetic modifications in the context of oncogenesis can de-repress HERV transcription and lead to translation of neoantigens \cite{Menendez2004, Wiesner2015}.

The cancer specificity of putative HERV peptides makes them promising biomarkers for and targets of immune therapies.
A major challenge to the development of HERV based therapies is the absence of consensus on how to find HERV gene products.
Proviruses are generally absent from human gene, transcript, and protein annotations, and their structural features make them hard to identify in agnostic surveys based on DNA or RNA sequencing \cite{ERVmap2018, Treangen2011}.
Proviral genomes are flanked by repetitive elements that often cannot be mapped with short read sequencing.
Furthermore, HERVs are transposable elements that have generated many similar proviral integrations throughout the human genome.
The multitude of similar sequences across the genome impairs accurate high throughout read alignment.
As a result, the genomic locations and transcriptional abundances of most HERVs are not known with confidence.

Recently, new approaches have been developed to overcome traditional limitations in HERV annotation, quantification, and protein identification \cite{Attig2017, Attig2019, ERVmap2018, Telescope2019}.
Here, we extend advances in the characterization of HERV transcriptomes to discover novel CRC specific expression patterns.
We test for differential expression of HERV transcripts in a large cohort of healthy, tumor-adjacent normal, and tumor RNA sequencing data and identify HERV transcripts over expressed in tumor samples.
We then predict the neoantigens most likely to succeed as biomarkers and immunogenic peptides in the treatment of CRC.

\section*{Methods}

We ...

\section*{Results}

We ...

\section*{Discussion}

We ...

\bibliography{main}
\bibliographystyle{vancouver}

\end{document}
