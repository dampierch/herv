\subsection*{Latent factor analysis}
To estimate the confounding influences of latent factors, including demographic variables with missing data, surrogate variable analysis was performed using the \emph{sva} package \citep{sva}.
First, the number of significant latent factors was estimated with the \verb|num.sv| function.
For this estimation, the Buja Eyuboglu method was used because it estimated a lower, more tractable number of latent factors.
The estimation was performed using a full model with two terms, one for the intercept and one for tissue phenotype.
Second, the latent factors (i.e. surrogate variables) themselves were estimated with the \verb|svaseq| function.
For this estimation, the same full model with intercept and phenotype terms was used, and the null model was set to include an intercept term alone.
The surrogate variables returned from this process included latent factors affecting every sample and were included as variables in the sample annotation matrix along with tissue phenotype.

\subsection*{Data vizualization}
Results of high dimensional data analysis were displayed in figures generated with the \emph{ggplot2} package \citep{Wickham2016}.
Multipanel figures were assembled using the \emph{cowplot} package \cite{Wilke2020}.

Expression levels used for vizualization were based on normalized, transformed, and adjusted counts.
Normalization was performed with the \emph{DESeq2} package \citep{Love2014} using the \verb|estimateSizeFactors| and then \verb|counts| functions.
The \verb|counts| function was used with the \verb|normalized| argument set to TRUE.
Transformation was performed with \emph{DESeq2} using the variance stabilizing transformation on normalized counts.
The variance stabilizing transformation includes a log2 transformation.
Finally, adjustment of normalized and transformed counts for latent factors was performed with \emph{limma} using the \verb|remoteBatchEffect| function.
The normalized, transformed, and adjusted counts were used in histograms to represent expression levels of GENCODE genes and HERV loci.
These counts were also overlaid atop boxplots to represent expression levels of tumor specific HERV loci.
Lastly, the same counts were used to calculate Z-scores to represent expression levels of all HERV loci tested.

Effect sizes (i.e. $log2(fold change)$) used for vizualization of correlations between different analyses were calculated by \emph{DESeq2} regression models and used without shrinkage.
Effect sizes used for volcano plots (i.e. scatter plots of $log2(fold change)$ versus $-log10(\emph{P})$) were shrunken with the \emph{ashr} package \citep{Stephens2020}.
