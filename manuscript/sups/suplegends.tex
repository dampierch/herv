\textbf{Figure 1.} Correlations of effect sizes across studies and cohorts. \\
a) Scatterplot demonstrating relationship between estimates of effect sizes (i.e. $log2(fold change)$) of the tumor phenotype on protein coding genes in the discovery cohort of this analysis (i.e. HERV Study) and Cohort A of a prior analysis of the same samples, as reported in Dampier et al. \citep{Dampier2020} (i.e. Field Effect Study).
Baseline expression is set by the healthy phenotype.
Only effect sizes on genes tested in both studies are shown.
\emph{R} = Pearson correlation coefficient. \\
b) Scatterplot demonstrating relationship between estimates of effect sizes (i.e. $log2(fold change)$) of the tumor phenotype on protein coding genes in the discovery cohort of this analysis (i.e. HERV Study) and Cohort A of a prior analysis of the same samples, as reported in Dampier et al. \citep{Dampier2020} (i.e. Field Effect Study).
Baseline expression is set by the tumor adjacent phenotype.
Only effect sizes on genes tested in both studies are shown. \\
c) Scatterplot demonstrating relationship between estimates of effect sizes (i.e. $log2(fold change)$) of the tumor phenotype on proviral loci in the discovery cohort of this analysis and independent cohort \#1.
Baseline expression is set by the healthy phenotype.
Only effect sizes on loci tested in both cohorts are shown. \\
d) Scatterplot demonstrating relationship between estimates of effect sizes (i.e. $log2(fold change)$) of the tumor phenotype on proviral loci in the discovery cohort of this analysis and independent cohort \#2.
Baseline expression is set by the tumor adjacent phenotype.
Only effect sizes on loci tested in both cohorts are shown.\\

\textbf{Figure 2.} Differential expression of unannotated human endogenous retroviral loci in colorectal transcriptomes using a novel HERV-K reference transcriptome. \\
Boxplots demonstrating expression levels by phenotype of six proviral loci associated with the tumor phenotype (\emph{P} adjusted < 0.05) in samples from the discovery cohort of this analysis analyzed using an alternative proviral reference transcriptome as described in Grabski et al. \citep{Grabski2020}.
Five of the six significant associations are negative.
